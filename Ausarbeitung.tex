% Header
\documentclass[12pt,a4paper,oneside,english]{article}
\documentclass[a4paper, 12pt]{article}%scrartcl

\usepackage[utf8]{inputenc}
\usepackage[T1]{fontenc}
\usepackage{lmodern}
\usepackage[german]{babel}
\usepackage{helvet}
\renewcommand{\familydefault}{\sfdefault}
\usepackage[onehalfspacing]{setspace}
\usepackage[a4paper, left=3cm, right=4cm, top=2cm]{geometry}
\usepackage{parskip}
\usepackage{bibgerm}
\usepackage{graphicx}
\usepackage{tcolorbox}
\usepackage{dingbat}
\usepackage{pdfpages}
\usepackage[hyphens]{url}
\usepackage{hyperref}
\usepackage{amsmath}
\usepackage{amsfonts}

\newcommand{\leadingzero}[1]{\ifnum #1<10 0\the#1\else\the#1\fi}
\newcommand{\datumVonHeute}{\leadingzero{\day}.\leadingzero{\month}.\the\year}

\hypersetup{
colorlinks=true,
urlcolor=blue,
}

% HIER EIGENE DATEN EINGEBEN!

\newcommand{\haThema}{CO2-Melder zur Gefahreneinschätzung des Infektionsrisikos des Coronavirus}
\newcommand{\haAutor}{Sandro Lobbene}
\newcommand{\haAutorAdresse}{Musterstr. 1}
\newcommand{\haAutorPLZ}{12345}
\newcommand{\haAutorOrt}{Musterhausen}
%\newcommand{\haDeckblattTextEins}{Examensarbeit für das zweite Staatsexamen\\ im Fach Berufliche Informatik für das Lehramt an berufsbildenden Schulen}
%\newcommand{\haDeckblattTextZwei}{Lehrkraft im Vorbereitungsdienst im 2. Semester}
\newcommand{\haSchule}{Gymnasium der Stadt Frechen}
\newcommand{\haSchuleAdresse}{Rotdornweg 43}
\newcommand{\haSchulePLZ}{50226}
\newcommand{\haSchuleOrt}{Frechen}
\newcommand{\haGutachter}{Oliver Dietershagen}
%\newcommand{\haGutachterText}{Studienleiter des IQSH für Berufliche Informatik}



\title{\haThema}
\author{\haAutor}
\date{\today}

\begin{document}

% DECKBLATT
\begin{titlepage}
\begin{center}
{\LARGE\bfseries \haThema \par}
\vfill
\begin{figure}[hbtp]
\begin{center}
\includegraphics[width=10cm]{img/rgb.jpg}
\label{imgDeckbild}
\end{center}
\end{figure}

Erstellt von:\\ {\bfseries \haAutor}\\ 
PH 1\\
2020/2021
%\haAutorAdresse\\ 
%\haAutorPLZ~\haAutorOrt 
\par
\vspace{1cm}
{\haSchule \\ \haSchuleAdresse \\ \haSchulePLZ~\haSchuleOrt \par}
\vspace{1cm}
Gutachter:\\ {\bfseries \haGutachter}
\vfill
\end{center}
%{\haAutorOrt,~\datumVonHeute\par}
\end{titlepage}
\thispagestyle{empty}
\newpage


% INHALTSVERZEICHNIS
\tableofcontents
\thispagestyle{empty}
\newpage

% Seitenzahl auf 1 setzen
\setcounter{page}{1}


% % % % % % % % % % % % % % % % % % % % % % % % % % %
% HIER KOMMT DER EIGENTLICHE INHALT DER HAUSARBEIT!

\section{Einleitung}


\newpage
\section{Allgemeine Funktion eines CO2-Melders}
\subsection {Problemstellung}

\begin{figure}[hbtp]
	\centering
	\includegraphics[width=15cm]{img/Infektionsverlauf.png}
	\caption{Infektionszahlen des vergangenen Zeitraums}
	\label{imgInfektionsverlauf}
\end{figure}


\subsection{CO2-Verbrauch bei Menschen}
\subsection{Messung der CO2-Werte}
\label{lblMessungderWerte}

\newpage
\section{Aufbau des Melders}

\newpage
\section{Funktionsweise des Sensors}
\label{lblFunktionsweisesensor}
\subsection{Produktdetails}
\paragraph{Funktion:}
\section{Zusammenfassung}
\newpage

\section{Literaturverzeichnis}
\subsection{Bücher}

[CHR20] Christian J. Kähler, Thomas Fuchs, Rainer Hain: Können mobile Raumluftreiniger eine indirekte SARS-CoV-2 Infektionsgefahr durch Aerosole wirksam reduzieren?, 2020

[NOR88] Norman, Don: The Design of everyday things, 1988

[UBA08] Umweltbundesamt: Gesundheitliche Bewertung von Kohlendioxid in der Innenraumluft, 2008
\subsection{Internetquellen}

[ACS] \url{https://pubs.acs.org/doi/full/10.1021/acs.est.9b04959} (Stand: \today) 
