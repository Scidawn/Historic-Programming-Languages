\documentclass[12pt,a4paper]{scrartcl}

\usepackage{amsmath,amssymb}
\usepackage[T1]{fontenc}    
\usepackage[utf8]{inputenc}
\usepackage[english]{babel}
\usepackage{graphicx}
\usepackage[a4paper, left=3cm, right=3cm, top=3cm]{geometry}
\usepackage[onehalfspacing]{setspace}
\usepackage{helvet}
\renewcommand{\familydefault}{\sfdefault}
\usepackage{lmodern}

%Gegebenenfalls koennen Titel, Author und Datum festgelegt werden
\title{Historic Programming Languages}
\author{Sandro Lobbene\\Toygun Ejder\\}
\date{\today}

\begin{document}
\maketitle

\section{Motivation}

\section{Fortran}
Fortran is a procedural programming language, which influenced many other programming languages. It was developed by IBM in the 1950s for scientific and engineering applications.[1] Fortran is a general-purpose, imperative programming language and plays a big role in numeric computation as well as scientific computing.\\
It is known for high-performance computing and is used for programs that benchmark and rank the world's fastest supercomputers.[2][4]\\
Fortran consists of numerous versions, which have added support for structured programming (FORTRAN 77), array programming, modular programming and generic programming (Fortran 90), high performance (Fortran 95), object-oriented programming since 2003 and concurrent programming in 2008.[1] Influenced languages by Fortran include BASIC, PL/I, and C, as well as Algol and Java.\\
Fortran is still used today for programming scientific and engineering applications. It is the primary language for some of the most intensive supercomputing tasks, such as weather forecasting, nuclear security and medicine. [3]\\


\section{References}
[1] https://en.wikipedia.org/wiki/Fortran
[2] https://en.wikibooks.org/wiki/Shelf:Fortran_programming_language
[3] https://gcn.com/cloud-infrastructure/2023/04/can-fortran-survive-another-15-years/385726/
[4] https://programmingforresearchers.leeds.ac.uk/fortran/

\end{document}
