% Header
\documentclass[12pt,a4paper,oneside,english]{article}

\usepackage[utf8]{inputenc}
\usepackage[T1]{fontenc}
\usepackage{lmodern}
\usepackage[english]{babel}
\usepackage{helvet}
\renewcommand{\familydefault}{\sfdefault}
\usepackage[onehalfspacing]{setspace}
\usepackage[a4paper, left=3cm, right=4cm, top=2cm]{geometry}
\usepackage{parskip}
\usepackage{bibgerm}
\usepackage{graphicx}
\usepackage{tcolorbox}
\usepackage{dingbat}
\usepackage{pdfpages}
\usepackage[hyphens]{url}
\usepackage{hyperref}
\usepackage{amsmath}
\usepackage{amsfonts}

\newcommand{\leadingzero}[1]{\ifnum #1<10 0\the#1\else\the#1\fi}
\newcommand{\datumVonHeute}{\leadingzero{\day}.\leadingzero{\month}.\the\year}

\hypersetup{
colorlinks=true,
urlcolor=blue,
}

% HIER EIGENE DATEN EINGEBEN!

\newcommand{\haThema}{Historic Programming Languages}
\newcommand{\haAutor}{Sandro Lobbene\\Toygun Ejder}
%\newcommand{\haDeckblattTextEins}{Examensarbeit für das zweite Staatsexamen\\ im Fach Berufliche Informatik für das Lehramt an berufsbildenden Schulen}
%\newcommand{\haDeckblattTextZwei}{Lehrkraft im Vorbereitungsdienst im 2. Semester}
\newcommand{\haSchule}{RWTH Aachen}
\newcommand{\haSchuleAdresse}{Templergraben 55}
\newcommand{\haSchulePLZ}{50226}
\newcommand{\haSchuleOrt}{Aachen}
\newcommand{\haGutachter}{Jürgen Giesl}


\title{\haThema}
\author{\haAutor}
\date{\today}

\begin{document}

% DECKBLATT
\begin{titlepage}
\begin{center}
{\LARGE\bfseries \haThema \par}
\vfill
\begin{figure}[hbtp]
\begin{center}
%\includegraphics[width=10cm]{img/rgb.jpg}
%\label{imgDeckbild}
\end{center}
\end{figure}

Erstellt von:\\ {\bfseries \haAutor}\\ 
 
\par
\vspace{1cm}
{\haSchule \\ \haSchuleAdresse \\ \haSchulePLZ~\haSchuleOrt \par}
\vspace{1cm}
Gutachter:\\ {\bfseries \haGutachter}
\vfill
\end{center}
%{\haAutorOrt,~\datumVonHeute\par}
\end{titlepage}
\thispagestyle{empty}
\newpage

% Seitenzahl auf 1 setzen
\setcounter{page}{1}

% % % % % % % % % % % % % % % % % % % % % % % % % % %
% HIER KOMMT DER EIGENTLICHE INHALT DER HAUSARBEIT!

\section{Introduction}


\newpage
\section{Allgemeine Funktion eines CO2-Melders}
\subsection {Problemstellung}

\subsection{CO2-Verbrauch bei Menschen}
\subsection{Messung der CO2-Werte}
\label{lblMessungderWerte}

\newpage
\section{Aufbau des Melders}

\newpage
\section{Funktionsweise des Sensors}
\label{lblFunktionsweisesensor}
\subsection{Produktdetails}
\paragraph{Funktion:}
\section{Zusammenfassung}
\newpage

\section{Literaturverzeichnis}
\subsection{Bücher}

[NOR88] Norman, Don: The Design of everyday things, 1988

[UBA08] Umweltbundesamt: Gesundheitliche Bewertung von Kohlendioxid in der Innenraumluft, 2008
\subsection{Internetquellen}

[ACS] \url{https://pubs.acs.org/doi/full/10.1021/acs.est.9b04959} (Stand: \today) 
